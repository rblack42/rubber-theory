\chapter{Preface}

For almost two decades, I taught a course titled \emph{Computer Architecture
and Machine Language}. The course was offered to second year college students
intending to become computer scientists. This course was by far my favorite
one, since it let me take students on an exciting journey of discovery.

When I first was offered this course, the basic material covered assembly
language programming for the Intel 8086 processor. Architecture took a back
seat to learning another programming language. I found that odd that the course
focused on a 16-bit processor since most computers at the time were using
32-bit Pentium processors. Upgrading the course required new development tools,
and I decided to make the course vendor neutral. That meant getting rid of the
Microsoft bias present in many courses at my school. I started off by deciding
to introduce my students to tools commonly found in open-source projects. Those
tools included programmer's editors, build tools, source code control tools,
and standards compliant compilers capable of building projects on any platform.
Many of my students were shocked to learn that they could build software
without using some magical \emph{Integrated Development Environment}. My theory
was simple. Choosing an IDE is something best left to that first job the
student will land. That job will have an development process and tool chain
that the new employee will need to learn. 

As the course evolved, I added more emphasis on the inner workings of the
machine the students were learning to program. At first the focus was still on
Intel chips since they power most of the computers students are familiar with.
However, the world is changing, and more and more work is being done on systems
that use other processors. The most common chips in mobile platforms today are
variants of the \emph{ARM} processor. 

The \emph{ARM} processors are complex chips, maybe too complex for beginners in
my course to study. There was a simple alternative available though. The
\emph{Microchip AVR} processor found on the new |emph{Arduino} development
boards was available very inexpensively. I decided to buy enough of these
boards to set up lab kits for my classes so students could do hands-on
development work on real hardware.

The course became very popular. It was challenging, but prepared students well
for their future jobs.

I continues to add material focusing on what was going on inside the processors
they were learning to control. Then in 2017 something interesting happened. The
Texas body charged with setting standards for college courses changed the
course requirements for my course in an interesting way. The new guidelines
asked students to write a simulator for a real processor as part of the course.
They also added a focus on embedded processors intending to get students ready
for those mobile platforms found everywhere. I was already doing most of what
they asked. I only needed to add the simulator to my course to meet these new
guidelines.

This book is designed for this course. Although I have now retired, I decided
to write this book based on my lecture notes but with a new twist.

Instead of producing yet another dry textbook, I decided to write a book along
the lines of one of my favorite books: \emph{Godel, Escher, Bach} by Douglas
Hofstadter \cite{Hofstadter:1999}. The result is not so much a textbook, but more
of a novel. I want the student to want to read this text, not just scan it
looking for answers to exercises they are given. I hope the result is
interesting enough to show them why they are learning all this new material. I
also hope to produce more professional candidates for that job market waiting
for them in their near future.

I hope you enjoy reading this book as much as I enjoyed producing it.

Roie R. Black  BS, MS Aerospace Engineering, MS Computer Science\\
Major, USAF (retired) \\
Professor, Computer Science (retired) \\
Austin Community College, Austin, Texas
